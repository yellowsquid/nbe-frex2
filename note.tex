% Created 2023-02-02 Thu 17:47
% Intended LaTeX compiler: xelatex
\documentclass[11pt,a4paper]{article}
\usepackage{longtable}
\usepackage{amsmath}
\usepackage{amssymb}
\usepackage{hyperref}
\usepackage[capitalise,noabbrev]{cleveref}
\usepackage{mathtools}
\usepackage{stmaryrd}
\usepackage{unicode-math}
\usepackage{parskip}
\usepackage{newunicodechar}
%% \usepackage[moderate]{savetrees}
\author{Greg Brown}
\date{\today}
\title{Frex\textsuperscript{2}: Normalisation by Evaluation for Second-Order Algebras}
\hypersetup{
 pdfauthor={Greg Brown},
 pdftitle={Frex\textsuperscript{2}: Normalisation by Evaluation for Second-Order Algebras},
 pdfkeywords={},
 pdfsubject={},
 pdfcreator={},
 pdfborder={0 0 0},
 pdflang={English}}

\newfontfamily{\fallbackfont}{STIX Two Math}[Scale=MatchLowercase]
\DeclareTextFontCommand\textfallback{\fallbackfont}

\setmathfont{STIX Two Math}

\newunicodechar{β}{\textfallback{β}}
\newunicodechar{ζ}{\textfallback{ζ}}
\newunicodechar{η}{\textfallback{η}}
\newunicodechar{λ}{\textfallback{λ}}
\newunicodechar{↝}{\textfallback{↝}}
\newunicodechar{↦}{\textfallback{↦}}
\newunicodechar{⟨}{\textfallback{⟨}}
\newunicodechar{⟩}{\textfallback{⟩}}

\newcommand*\ctxext[2]{{#1}, {#2}}
\newcommand*\ctxcons[2]{{#1}, {#2}}
\newcommand*\ctxempty[0]{\emptyset}
\newcommand*\sta[1]{\mathop{\overline{#1}}}
\newcommand*\dyn[1]{\mathop{\underline{#1}}}
\newcommand*\Lapp[2]{\texttt{\ensuremath{#1} \$ \ensuremath{#2}}}
\newcommand*\Labs[2]{\texttt{λ{#1}. \ensuremath{#2}}}
\newcommand*\msub[2]{\texttt{\ensuremath{#1}⟨\ensuremath{#2}⟩}}

\begin{document}

\maketitle

\section{Second-Order Algebra}

Universal second-order algebra is a technique to describe syntax and theories
for algebras with operations that can bind variables. One notable example of a
second-order algebra is the simply-typed lambda calculus (STLC). Second-order
algebras depend on a set \(T\) of types. For instance, the grammar \(T \coloneqq
{N \mid {T \rightarrowtail T}}\) gives the types for the STLC, where \(N\) describes some neutral
base type.

A \emph{signature} is a pair of a set \(O\) and function \({\vert\cdot\vert} \in {O \to {{(T^* \times
T)}^* \times T}}\). The set \(O\) is the set of operators in the signature. The
function gives the arity and type of each operator. Because we want operators to
bind variables, the arguments are a pair of the list of types of bound
variables, and the type of the argument. Let the notation \({(\gamma_i)\sigma_i} \vdash o : \tau\)
denote \({\vert o \vert} = {{(\gamma_i , \sigma_i)}_i , \tau}\).

The signature for the STLC has the two operators \(\mathtt{app}^{\alpha,\beta}\) and
\(\mathtt{abs}^{\alpha,\beta}\) for each pair of types \(\alpha\) and \(\beta\). Their arities are
\({{\alpha \rightarrowtail \beta}, \alpha} \vdash \mathtt{app}^{\alpha,\beta} : \beta\) and \({(\alpha)\beta} \vdash \mathtt{abs}^{\alpha,\beta} : {\alpha
\rightarrowtail \beta}\).  Application takes two arguments---a function and value---and returns a
result. Neither argument to application binds variables. Abstraction takes a
single argument, of the output type of the newly-defined function. This argument
has access to a freshly-bound variable of the input type to the function.

The unpairing operation in STLC with products is also a second-order operation.
The operation \verb|let (x, y) = e in e'| is represented by an operator
\(\mathtt{unpair}^{\alpha,\beta,\tau}\), with \({{\alpha \times \beta}, {(\alpha, \beta)\tau}} \vdash
\mathtt{unpair}^{\alpha,\beta,\tau} : \tau\). The argument \verb|e| is a pair and has no
additional variables in scope. The second argument \verb|e'| has access to two
new variables, corresponding to the two components of the pair.

A non-example of a second-order operation is non-linear pattern matching, where
the same variable appears twice in the pattern. One example is the pattern
\verb|let (x, x) = e in e'|.

Given a signature \((O , {\vert\cdot\vert})\), you can define an \emph{algebra}. Algebras
have four constituent parts:

\begin{itemize}
\item a carrier sorted family \(\mathcal{A} = {T \to T^* \to \mathbf{Set}}\)
\item an operation \({\llbracket\cdot\rrbracket}_o \in
  {{(\mathcal{A}\,\sigma_i\,(\ctxext{\gamma_i}{\Gamma}))}_i \to {\mathcal{A}\,\tau\,\Gamma}}\) for each operator
  \({(\gamma_i)\sigma_i} \vdash o : \tau\) and context \(\Gamma \in {T^*}\)
\item a mapping \(\mathtt{var} \in {{\mathcal{I}\,\tau\,\Gamma} \to {\mathcal{A}\,\tau\,\Gamma}}\)
\item a mapping \(\mathtt{sub} \in {{\mathcal{A}\,\tau\,\Gamma} \to (\forall \sigma. {{\mathcal{I}\,\sigma\,\Gamma} \to {\mathcal{A}\,\sigma\,\Delta}}) \to
  {\mathcal{A}\,\tau\,\Delta}}\)
\end{itemize}

where \(\mathcal{I}\) is the sorted family of variables.

Write \(\Gamma \vdash^{\mathcal{X}} \mathtt{t} : \tau\) for \(\mathtt{t} \in {\mathcal{X}\,\tau\,\Gamma}\), for any sorted
family \(\mathcal{X}\).

The operations, \(\mathtt{var}\) and \(\mathtt{sub}\) must obey a collection of
coherence conditions. These assert that \(\mathtt{var}\) and \(\mathtt{sub}\)
form a substitution monoid, and that \(\mathtt{sub}\) commutes with the
operations in a capture-avoiding way\footnotemark{}.

\footnotetext{What are they exactly?}

For the STLC, an algebra \(\mathcal{A}\) has a carrier \(\mathcal{A}\); a pair of operations
\({\llbracket\cdot\rrbracket}_\mathtt{app} \in {{\mathcal{A}\,(\alpha\rightarrowtail\beta)\,\Gamma} \to {\mathcal{A}\,\alpha\,\Gamma} \to
  {\mathcal{A}\,\beta\,\Gamma}}\), and \({\llbracket\cdot\rrbracket}_\mathtt{abs} \in
  {{\mathcal{A}\,\beta\,(\ctxcons{\alpha}{\Gamma})} \to {\mathcal{A}\,(\alpha\rightarrowtail\beta)\,\Gamma}}\); and the variable and
  substitution maps.

One example algebra for any signature is the unit algebra \(\mathbf{1}\), where
\(\mathbf{1}\,\tau\,\Gamma = {\{\ast\}}\).

Another example is the algebra of sets and functions for the STLC.\@ First, take
any set \(V\) to represent values of type \(N\). A type \(T\) can then be
interpreted as a set of values \(\mathcal{V}[T]\):

\begin{align*}
{\mathcal{V}[N]} &\mapsto V \\
{\mathcal{V}[\alpha \rightarrowtail \beta]} &\mapsto {{\mathcal{V}[\alpha]} \to {\mathcal{V}[\beta]}}
\end{align*}

We can take the carrier sorted family \({\mathcal{A}\,\tau\,\Gamma} = {(\forall \sigma. {{\mathcal{I}\,\sigma\,\Gamma} \to
{\mathcal{V}[\sigma]}}) \to {\mathcal{V}[\tau]}}\). From here, defining the operations and mappings
is straight forward:

\begin{align*}
{{\llbracket {f , g} \rrbracket}_{\mathtt{app}}\,\gamma} &= {f\,\gamma\,(g\,\gamma)} \\
{{\llbracket {f} \rrbracket}_{\mathtt{abs}}\,\gamma} &=
  {x \mapsto {f\,({\mathtt{here} \mapsto x}; {{\mathtt{there}\,v} \mapsto \gamma\,v})}} \\
{\mathtt{var}\,v\,\gamma} &= \gamma\,v \\
{\mathtt{sub}\,f\,\sigma\,\gamma} &= {f\,(v \mapsto {\sigma\,v\,\gamma})}
\end{align*}

Verifying that this definition satisfies the coherence conditions is left as an
exercise\footnotemark{}.

\footnotetext{This is me being lazy.}

\section{Metavariables}

Here is a typical of writing the \(\beta\)-reduction relation for STLC:\@

\begin{verbatim}
(λ x. t) $ u ↝ t[x ↦ u]
\end{verbatim}

The variable \verb|t| is implicitly parameterised by the bound variable
\verb|x|, meaning \({\ctxcons{\mathtt{x} : \alpha}{\Gamma}} \vdash \mathtt{t} : \beta\). We can
make this dependency explicit. Similarly, \verb|u| is not parameterised by any
terms:

\begin{verbatim}
(λ x. t<x>) $ u<> ↝ t<u<>>
\end{verbatim}

We can say that \verb|t| and \verb|u| are \emph{metavariables}---free variables
with parameters. We can instantiate metavariables with concrete terms to create
new ones. This property is formalised in the definition of a \emph{syntactic
algebra}.

A syntactic algebra for a signature \(\Sigma\) over a sorted family of metavariables
\(\mathfrak{X}\) is:

\begin{itemize}
\item an algebra \(\mathcal{A}\) over the signature
\item a map \(\mathtt{mvar} \in {{\mathfrak{X}\,\tau\,\Pi} \to (\forall \sigma. {{\mathcal{I}\,\sigma\,\Pi} \to {\mathcal{A}\,\sigma\,\Gamma}}) \to
  {\mathcal{A}\,\tau\,\Gamma}}\)
\item and a map \(\mathtt{msub} \in {{\mathcal{A}\,\tau\,\Gamma} \to (\forall {\sigma, \Theta}. {{\mathfrak{X}\,\sigma\,\Theta} \to
  {\mathcal{A}\,\sigma\,(\ctxext{\Theta}{\Delta})}}) \to {\mathcal{A}\,\tau\,(\ctxext{\Gamma}{\Delta})}}\)
\end{itemize}

where \(\mathtt{mvar}\) and \(\mathtt{msub}\) are subject to conditions meaning
they behave well with each other and \(\mathtt{sub}\)\footnotemark{}.

\footnotetext{At some point I \emph{need} to give these conditions. Not yet.}

The map \(\mathtt{mvar}\) takes a metavariable \(\Pi \vdash \mathfrak{m} : \tau\) and converts it
into a term in context \(\Gamma\). This is by giving a value for each of \(\mathfrak{m}\)'s
parameters, achieved via the substitution argument.

The syntactic substitution map \(\mathtt{msub}\) is more complex. First note
that it is linear: the output context is an extension of the input. This means
that metasubstitution can use bound variables, as long as they were bound in a
higher scope. Also observe that the substitution map (the second parameter) is
independant of the input context \(\Gamma\). Because metavariables are always
associated with a substitution, it is sufficient to map only the metavariables
in their original contexts. The context can be ``corrected'' by a substitution.

Note that the term algebra \(\mathbb{T}[\mathfrak{X}]\) for a signature is the free syntactic
algebra over \(\mathfrak{X}\), with for each map \(f \in {{\mathfrak{X}\,\tau\,\Gamma}\to{\mathcal{A}\,\tau\,\Gamma}}\) a
unique homomorphism \(\mathtt{bind}(f) \in {{\mathbb{T}[\mathfrak{X}]} \to \mathcal{A}}\).

Return to the earlier example of \(\beta\) reduction. We have
\(\ctxcons{\alpha}{\ctxempty} \vdash \mathtt{t} : \beta\) and \(\ctxempty \vdash \mathtt{u} : \alpha\)
as our metavariables \(\mathfrak{X}\). We then have two terms, both of type \(\beta\) in an
empty context: \verb|(λ x. t<x>) $ u<>| and \verb|t<u<>>|.

Working in a new context \(\Gamma =
{\ctxcons{\mathtt{f} : {\alpha\rightarrowtail\beta}}{\ctxcons{\mathtt{x} : \alpha}{\ctxempty}}}\), we can construct
the syntactic substitution map \(\zeta \in {{\mathfrak{X}\,\sigma\,\Theta} \to {\mathcal{A}\,\sigma\,(\ctxext{\Theta}{\Gamma})}}\):

\begin{align*}
  \ctxcons{\mathtt{y} : \alpha}{\Gamma} \vdash {\zeta(\mathtt{t})} &= \Lapp{\mathtt{f}}{\mathtt{y}}
  : \beta \\ \Gamma \vdash {\zeta(\mathtt{u})} &= x : \alpha
\end{align*}

where \(\mathtt{y}\) in the first equation is the parameter of metavariable
\(\mathtt{t}\).

An example application is \verb|(t<u<>>)<ζ>|, which is equivalent to \verb|f $ x|.

\section{Free Extensions}

Given an algebra \(\mathcal{A}\) and metavariables \(\mathfrak{X}\), a \emph{free extension} of \(\mathcal{A}\)
by \(\mathfrak{X}\) is a syntactic algebra \(\mathcal{F}\) over \(\mathfrak{X}\) with the following additional
structure:

\begin{itemize}
\item a homomorphism \(\mathtt{sta} \in {\mathcal{A}\to\mathcal{F}}\)
\item for every pair of homomorphism \(f \in {\mathcal{A}\to\mathcal{B}}\) and map \(g \in
  {{\mathcal{X}\,\tau\,\Gamma}\to{\mathcal{B}\,\tau\,\Gamma}}\), a homomorphism \(\mathtt{interp}(f,g) \in {\mathcal{F}\to\mathcal{B}}\)
\end{itemize}

such that things play nicely\footnotemark{}.

\footnotetext{I am still resisting giving conditions.}

The free extension is an extension in the sense that the algebra \(\mathcal{A}\) is given
an interpretation of metavariables. It is free in the sense that adding the
metavariables does not restrict the homomorphisms to or from \(\mathcal{A}\).

Define the map \(\mathtt{dyn} \in {{\mathfrak{X}\,\tau\,\Gamma}\to{\mathcal{F}\,\tau\,\Gamma}}\) by \(\mathtt{dyn}(\mathfrak{m}) =
\mathtt{msub}(\mathfrak{m},\mathtt{var})\). The names \(\mathtt{sta}\) and \(\mathtt{dyn}\)
come from the perspective of staged compilation: values made with
\(\mathtt{sta}\) are known statically and can be acted on during compilation.
Values defined via \(\mathtt{dyn}\) are only known at runtime and have no
inherent algebraic structure.

One universal instance of the free extension is given by \(\mathbb{T}{[\mathcal{A} \cup \mathfrak{X}]}\), quotient
the conditions listed earlier. Define homomorphisms \(\mathtt{eval} \in
{{\mathbb{T}{[\mathcal{A} \cup \mathfrak{X}]}} \to \mathcal{F}}\) and \(\mathtt{reify} \in {\mathcal{F} \to {\mathbb{T}{[\mathcal{A} \cup \mathfrak{X}]}}}\) given by
\(\mathtt{bind}\) and \(\mathtt{interp}\) respectively. \(\mathtt{eval}\)
ideally performs static computations, normalising the term in some way, and
\(\mathtt{reify}\) exports the normal form as a term.

For brevity, let \(\sta{\mathtt{t}}\) represent \(\mathtt{sta}(\mathtt{t})\) and
understand \(\dyn{\mathfrak{m}}\) to be \(\mathtt{dyn}(\mathfrak{m})\).

\section{βη-Normal Form as a Free Extension}

In the STLC, βη-normal form is a way of writing terms such that test for βη
equivalence is a simple identity check. I will adapt this normal form to attempt
to produce a free extension of a STLC model \(\mathcal{A}\) by metavariables \(\mathfrak{X}\).

There are a couple of minor conceptual optimisations that we can apply before
evaluating terms. Remember that \(\mathtt{sta}\) is a homomorphism, thus
\(\mathtt{x}_i = \sta{\mathtt{x}_i}\) for all variables \(\mathtt{x}_i\). By
applying this as a substitutino on terms, we only have to consider application,
abstraction, and the static and dynamic metavariables. Further, we write
\(\sta{\mathtt{t}}\) for \(\msub{\sta{\mathtt{t}}}{\mathtt{var}}\), and treat it
as a distinct case.

Define a term to be \emph{dynamic} if there is a subterm that uses a dynamic
metavariable. For example, \(\msub{\dyn{\mathtt{m}}}{\zeta}\) and
\(\Lapp{\sta{\mathtt{t}}}{(\Labs{x}{\msub{\dyn{\mathtt{n}}}{\mathtt{x}}})}\) are
both dynamic, whilst \(\sta{\mathtt{t}}\) and
\(\Labs{x}{\msub{\sta{\mathtt{u}}}{\mathtt{x}}}\) are neither dynamic
(henceforth \emph{static}).

Here is where I define the normal form\footnotemark{}.

\footnotetext{I should do this.}

\section{The Equality Problem}

Unfortunately, the normal form just presented is not a model of STLC.\@ In fact,
it is not even an algebra. The problem comes from substitution, and can be
demonstrated by the following two equal terms:

\begin{align*}
\msub{\sta{\Lapp{\mathtt{x}_i}{\mathtt{t}}}}{\dyn{\zeta}}
  &= \msub{(\Lapp{\sta{\mathtt{x}_i}}{\sta{\mathtt{t}}})}{\dyn{\zeta}} \\
  &= \msub{(\Lapp{\mathtt{x}_i}{\sta{\mathtt{t}}})}{\dyn{\zeta}} \\
  &= \Lapp{\msub{\mathtt{x}_i}{\dyn{\zeta}}}{\msub{\sta{\mathtt{t}}}{\dyn{\zeta}}} \\
  &= \Lapp{\dyn{\zeta_i}}{\msub{\sta{\mathtt{t}}}{\dyn{\zeta}}}
\end{align*}

Assuming the terms have type \(N\), then they are in normal form. The top two
lines normalise to the first term, and the bottom two lines normalise to the
final term. However, because the subterm
\(\sta{\Lapp{\mathtt{x}_i}{\mathtt{t}}}\) is opaque for an arbitrary algebra
\(\mathcal{A}\), these two normal forms for the same term are necessarily distinct.

For this normal form to be an algebra, it must be taken as a quotient with
respect to some relation\footnotemark{}.

\footnotetext{I need to work out what exactly.}

\section{Removing the Quotient}

This is full of loose ideas, without any proof.

The quotient over the normal form can be eliminated in some cases. Starting
simple, the quotient is eliminable when the algebra \(\mathcal{A}\) to be extend by \(\mathfrak{X}\)
is given by \(\mathcal{A} = {\mathbb{T}[\mathfrak{Y}]}\) for some fixed family of metavariables \(\mathfrak{Y}\). The
presented normal form is also the βη-normal form of \(\mathbb{T}[\mathfrak{X} \cup \mathfrak{Y}]\), which is
an algebra without needing a quotient.

Another example are algebras \(\mathcal{A}\) that are ``normal forms'', with a
\(\mathfrak{Y}\)-syntactic-algebra homomorphism \(\mathcal{A} \to {\mathbb{T}[\mathfrak{Y}]}\) for some fixed family \(\mathfrak{Y}\).
To evaluate terms \(\mathbb{T}[\mathcal{A} \cup \mathfrak{X}]\), first forget \(\mathcal{A}\) to obtain \(\mathbb{T}[{\mathbb{T}[\mathcal{A}]} \cup \mathfrak{X}]\).
Next normalise this as above, and recompute \(\mathcal{A}\) using \(\mathtt{interp}\) and
\(\mathtt{bind}\).

\end{document}

% LocalWords:  unpair/pair
